\documentclass{article}
\usepackage[table]{xcolor}
%\usepackage[showframe=true]{geometry}
\usepackage{changepage}
\usepackage{fullpage} % Package to use full page
\usepackage{parskip} % Package to tweak paragraph skipping
\usepackage{tikz} % Package for drawing
\usepackage{amsmath}
\usepackage{hyperref}
\usepackage{amsmath,amssymb}
\usepackage{color}
\usepackage[version=4]{mhchem} 
\usepackage{bm}
\usepackage{verbatim}
\usepackage{subfig}

\newcommand{\head}[1]{\textnormal{\textbf{#1}}}
\setlength{\arrayrulewidth}{1mm}
\setlength{\tabcolsep}{8pt}
\renewcommand{\arraystretch}{1.75}

\begin{document}
	{\rowcolors{2}{gray!20}{white}	
	\begin{table}[!h]
		\begin{tabular}{||c|c|c||}
			\hline
			\rowcolor{orange!85!black} \textcolor{white}{\head{Identity}} & \textcolor{white}{\head{Equation}} & \textcolor{white}{\head{Property}} \\
			\hline
			Inner product & $\mathbf{a} \cdot \mathbf{b} =\frac{1}{2}(\mathbf{ab}+\mathbf{ba})=\mathbf{b}\cdot \mathbf{a}$ & Symmetric\\
            Wedge product & $\mathbf{a} \wedge \mathbf{b} =\frac{1}{2}(\mathbf{ab}-\mathbf{ba})=\mathbf{b}\cdot \mathbf{a}$ & Anti-symmetric\\
            Geometric product & $\mathbf{ab} = \mathbf{a} \cdot \mathbf{b} + \mathbf{a} \wedge \mathbf{b} $ & \begin{tabular}{c}
                 Canonical decomposition of \\
                 inner and outer products
            \end{tabular} \\
			\hline
		\end{tabular}
	\end{table}}
\end{document}