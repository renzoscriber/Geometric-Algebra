\documentclass[a4paper]{article}
% Preamble
\usepackage[utf8]{inputenc}
\usepackage{fullpage}
\usepackage[english]{babel}
\usepackage{color}
\usepackage{amsmath}
\usepackage{url}
\usepackage{standalone}
\usepackage{parskip}
\usepackage{graphicx}
\usepackage{caption}
\usepackage{subcaption}
\usepackage{natbib}
\usepackage{amsfonts}

\title{Geometric Algebra}
\author{Lauren Shriver}
\date{Fall 2019}

\begin{document}
\maketitle

\section*{The Geometric Product in $G_3$}
Preliminary identities: $\mathbf{ab}$ for vectors $\mathbf{a}$, $\mathbf{b}$, and $\mathbf{c}$:
    \begin{equation}
        (\mathbf{ab})\mathbf{c} = \mathbf{a}(\mathbf{bc}), \quad \mathrm{Associative}
    \end{equation}
    \begin{equation}
        \mathbf{a}(\mathbf{b}+\mathbf{c}) = \mathbf{ab}+\mathbf{ac}, \quad \mathrm{Left \, \, Distributive}
    \end{equation}
    \begin{equation}
        (\mathbf{b}+\mathbf{c})\mathbf{a} = \mathbf{ba}+\mathbf{ca}, \quad \mathrm{Right \, \, Distributive}
    \end{equation}
    \begin{equation}
        \mathbf{a}^2 = |\mathbf{a}|^2, \quad \mathrm{Contraction}
    \end{equation}
\subsection*{The vector-vector geometric product}
    \begin{equation}
        \mathbf{u}\mathbf{v} = \mathbf{u}\cdot\mathbf{v} + \mathbf{u} \wedge \mathbf{v}
    \end{equation}
    \begin{equation}
        \mathbf{u} \cdot \mathbf{v} = \frac{1}{2} (\mathbf{u}\mathbf{v} + \mathbf{v}\mathbf{u}) = \langle \mathbf{u}\mathbf{v} \rangle_0
    \end{equation}
        \begin{equation}
        \mathbf{u} \wedge \mathbf{v} = \frac{1}{2} (\mathbf{u}\mathbf{v} - \mathbf{v}\mathbf{u}) = \langle \mathbf{u}\mathbf{v} \rangle_2
    \end{equation}
\subsection*{The vector-bivector geometric product}
        \begin{equation}
        \mathbf{a}\mathbf{B} = \mathbf{a}\cdot\mathbf{B} + \mathbf{a} \wedge \mathbf{B}
    \end{equation}
    \begin{equation}
        \mathbf{a} \cdot \mathbf{B} = \frac{1}{2} (\mathbf{a}\mathbf{B} - \mathbf{B}\mathbf{a}) = \langle \mathbf{a}\mathbf{B} \rangle_1
    \end{equation}
        \begin{equation}
        \mathbf{a} \wedge \mathbf{B} = \frac{1}{2} (\mathbf{a}\mathbf{B} + \mathbf{B}\mathbf{a}) = \langle \mathbf{a}\mathbf{B} \rangle_3
    \end{equation}
\section*{Dual and the Cross Product}
General expressions relating a bivector $\mathbf{B} = \mathbf{u} \wedge \mathbf{v}$ to its corresponding dual vector:
    \begin{equation}
        \mathbf{u} \times \mathbf{v} = (\mathbf{u} \wedge \mathbf{v})I^{-1} = \mathbf{B}I^{-1}
    \end{equation}
    \begin{equation}
        \mathbf{B} = \mathbf{u} \wedge \mathbf{v} = (\mathbf{u} \times \mathbf{v})I
    \end{equation}
\subsection*{Scalar-triple product}
For $\mathbf{u} \wedge \mathbf{v} \wedge \mathbf{w} = \lambda I$, we get
    \begin{equation}
        \lambda = \mathbf{u} \cdot (\mathbf{v} \times \mathbf{w})
    \end{equation}
\subsection*{Vector-triple product}
    \begin{equation}
        \mathbf{u} \times (\mathbf{v} \times \mathbf{w}) = \mathbf{u} \cdot (\mathbf{w} \wedge \mathbf{v})
    \end{equation}
\section*{Linear Transformations and Outermorphsisms}
\subsection*{Background}
Two requirements for a multi-variable function $f$ to be considered a linear transformation:
    \begin{enumerate}
        \item $f(\vec{x} + \vec{y}) = f(\vec{x}) + f(\vec{y})$
        \item $f(\lambda\vec{x}) = \lambda f(\vec{x})$
    \end{enumerate}
We can unify these two requirements into a single equation:
    \begin{equation}
        f(\alpha \vec{x} + \beta \vec{y}) = \alpha f(\vec{x}) + \beta f(\vec{y})
    \end{equation}
\subsubsection*{Combining Linear Transformations}
If $f$ is a linear transformation, then a scalar multiple of $f$ is also a linear transformation:
    \begin{equation}
        (\lambda f)(\vec{x}) = \lambda f(\vec{x})
    \end{equation}
The sum of two linear transformations is also a linear transformation:
    \begin{equation}
        (f+g)(\vec{x}) = f(\vec{x}) + g(\vec{x})
    \end{equation}
The composition of two linear transformations is also a linear transformation:
    \begin{equation}
        (fg)(\vec{x}) = f(g(\vec{x}))
    \end{equation}
Linear transformations are associative:
    \begin{equation}
        f(gh) = (fg)h
    \end{equation}
Generally, linear transformations are NOT necessarily commutative:
    \begin{equation}
        fg \neq gf
    \end{equation}
Linear transformations are distributive:
    \begin{equation}
        f(g+h) = fg + fh
    \end{equation}
\subsection*{Outermorphisms: Extending Linear Algebra}
Note: Outermorphisms preserve the structure of the outer (i.e., wedge) product 
\subsubsection*{LTs of Bivectors}
A linear transformation/outermorphism of a bivector (i.e., an oriented patch of area) returns another bivector
\\
\\
General rule:
    \begin{equation}
        f(\vec{u}\wedge \vec{v}) = f(\vec{u}) \wedge f(\vec{v})
    \end{equation}
General rule for a linear transformation acting on a sum of blades:
    \begin{equation}
        f(\vec{a} \wedge \vec{b} + \vec{c} \wedge \vec{d}) = f(\vec{a} \wedge \vec{b}) + f(\vec{c} \wedge \vec{d})
    \end{equation}
\subsubsection*{LTs of Trivectors}
A linear transformation/outermorphism of a trivector (i.e., an oriented patch of volume) returns another trivector
\\
\\
General rules:
    \begin{equation}
        f(\vec{u} \wedge \vec{v} \wedge \vec{w}) = f(\vec{u}) \wedge f(\vec{v}) \wedge f(\vec{w})
    \end{equation}
    \begin{equation}
        f(\lambda(\vec{u} \wedge \vec{v} \wedge \vec{w})) = \lambda(f(\vec{u}) \wedge f(\vec{v}) \wedge f(\vec{w}))
    \end{equation}
\end{document}
